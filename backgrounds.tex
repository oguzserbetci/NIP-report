Measuring risk and peoples behaviour under risky choice is a much debated topic in economics. Today thee major theories have emerged, that we will quickly introduce. In purely financial situations often variance of an asset is used as a proxy for risk. One calculates the variance and expected outcome of each option and then picks the one with highest outcome and smallest risk. A bit more sophisticated measure are Value At Risk, semi-variance (splitting variance into losses and gains) and expected shortfall. But all of these are higly abstract and requiere a person to sit down and calculated the numbers. 
% http://viking.som.yale.edu/will/hedge/Risk_BobJaeger.pdf
This hardly describes peoples everyday risk behaviour, instead a simpler and more intuitive way is requiered.
% add example

A method addressing these and more concerns with mean variance analysis has been proposed in the 1940s by Von Neumann and Morgenstern. They postulate that when facing a risky decision people will maximise not their expected value but their expected utility.  
To every possible outcome they assign a utility value as defined by a real valued utility function. The person does not need to know (and most likely doesn't know) about their utility function. They will implicitly assign utilities and then pick the option that maximizes their expected function value.

To illustrate this we will look at the following example:
