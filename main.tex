%%%%%%%% ICML 2018 EXAMPLE LATEX SUBMISSION FILE %%%%%%%%%%%%%%%%%

\documentclass{article}

\usepackage{amsmath}
% Recommended, but optional, packages for figures and better typesetting:
\usepackage{amstext}
\usepackage{microtype}
\usepackage{graphicx}
\usepackage{subfigure}
\usepackage{tikz}
\usepackage{booktabs} % for professional tables

\usetikzlibrary {positioning}
\usepackage{graphicx,caption}

\usepackage{subcaption}
\definecolor {processblue}{cmyk}{0.96,0,0,0}

\usepackage[T1]{fontenc} 
\usepackage[utf8]{inputenc}
\pagenumbering{roman}

% hyperref makes hyperlinks in the resulting PDF.
% If your build breaks (sometimes temporarily if a hyperlink spans a page)
% please comment out the following usepackage line and replace
% \usepackage{icml2018} with \usepackage[nohyperref]{icml2018} above.
\usepackage{hyperref}

% Attempt to make hyperref and algorithmic work together better:
\newcommand{\theHalgorithm}{\arabic{algorithm}}

% Use the following line for the initial blind version submitted for review:
% \usepackage{icml2018}

% If accepted, instead use the following line for the camera-ready submission:
\usepackage[accepted]{icml2018}

% The \icmltitle you define below is probably too long as a header.
% Therefore, a short form for the running title is supplied here:
\icmltitlerunning{Neural mechanisms of risk-sensitive choice and reinforcement
learning under uncertainty}

\newcommand{\keyword}[1]{\textit{#1}}

\begin{document}

\twocolumn[
\icmltitle{Neural mechanisms of risk-sensitive choice and reinforcement
learning under uncertainty}

% It is OKAY to include author information, even for blind
% submissions: the style file will automatically remove it for you
% unless you've provided the [accepted] option to the icml2018
% package.

% List of affiliations: The first argument should be a (short)
% identifier you will use later to specify author affiliations
% Academic affiliations should list Department, University, City, Region, Country
% Industry affiliations should list Company, City, Region, Country

% You can specify symbols, otherwise they are numbered in order.
% Ideally, you should not use this facility. Affiliations will be numbered
% in order of appearance and this is the preferred way.

\begin{icmlauthorlist}
\icmlauthor{Emil Azadian}{equal,tu}
\icmlauthor{Jan Botsch}{equal,tu}
\icmlauthor{Vlad-Catalin Frasineanu}{equal,tu}
\icmlauthor{Oğuz Şerbetci}{equal,tu}
\icmlauthor{Stephan Tietz}{equal,tu}
\end{icmlauthorlist}

\icmlaffiliation{tu}{TU Berlin, Neural Information Processing Group}

\icmlcorrespondingauthor{Emil Azadian}{emil.azadian@campus.tu-berlin.de}
\icmlcorrespondingauthor{Jan Botsch}{jan.botsch@campus.tu-berlin.de}
\icmlcorrespondingauthor{Vlad-Catalin Frasineanu}{v.frasineanu@campus.tu-berlin.de}
\icmlcorrespondingauthor{Oğuz Şerbetci}{oguz.serbetci@campus.tu-berlin.de}
\icmlcorrespondingauthor{Stephan Tietz}{stietz@campus.tu-berlin.de}

% You may provide any keywords that you
% find helpful for describing your paper; these are used to populate
% the "keywords" metadata in the PDF but will not be shown in the document
\icmlkeywords{Machine Learning, Reinforcement Learning, Risk Sensitivity}

\vskip 0.3in
]

% this must go after the closing bracket ] following \twocolumn[ ...

% This command actually creates the footnote in the first column
% listing the affiliations and the copyright notice.
% The command takes one argument, which is text to display at the start of the footnote.
% The \icmlEqualContribution command is standard text for equal contribution.
% Remove it (just {}) if you do not need this facility.

% \printAffiliationsAndNotice{}  % leave blank if no need to mention equal contribution
\printAffiliationsAndNotice{\icmlEqualContribution} % otherwise use the standard text.

\begin{abstract}
\textit{Utility functions are often used to explain human risk behaviour under uncertainty. We present a class of reinforcement learning agents trained on different utility functions that are able to mimic risky choice under uncertainty. We use self-gathered experimental data capturing risk profiles of human participants and apply the agent's results to them. From that, we deduct that the widely used exponential utility function might not be able to explain all human risk profiles. In turn, we present two other utility functions that could be suited for mapping risk profiles.}
\end{abstract}

\section{Introduction}\label{sec:introduction}
In our everyday lives we are forced to act under uncertainty, meaning in circumstances where we do not have full information. Imagine for example that you want to catch the local bus to university. You know when it is supposed to arrive, but it could very well be that the bus is a couple of minutes early or late - there is uncertainty involved. How we act under uncertainty is an indicator of our risk profile: Some people will want to avoid the risk of missing the bus and decide to leave for the station rather early, others might speculate that the bus will be late and leave accordingly, expressing a more risky behaviour. This behaviour is called the risk profile or risk-sensitivity of a person.\\
It has long been a goal of economics to capture risk profiles in a consistent manner. While it's easy to simulate behaviour for a given policy the reverse problem is yet to be solved.
In this paper we will show a machine learning approach that can be used to bridge this gap from observing peoples behaviour and estimating their utility function.
Based on this approach we will reassess the hypothesis that the so called exponential utility function models humans behaviour.
\section{Background}\label{sec:background}
\subsection{Measuring risk}

Measuring risk and peoples behaviour under risky choice is a much debated topic in economics. In purely financial contexts variance is often used as a proxy for risk. Options with high variance are considered more risky than options with lower variance. 
A decision maker would therefore calculate the expected return and the corresponding variance of each option available.
Then if one option has lower risk and higher expected return than the other it is said to be dominant and should be preferred.
This model has some shortfalls, i.e. it is easy to construct cases where empirical studies clearly show that it is not sufficient to explain human behaviour.
% More complex measures have been introduced to explain these behaviours like Value At Risk, semi-variance (splitting variance into losses and gains) and expected shortfall. But all of these are highly abstract and would require quite some effort to evaluate as variance is hard to estimate. 
It can therefore be assumed that this hardly describes peoples everyday risk behaviour, instead a simpler and more intuitive way is required. \cite{Jaeger00}
% add example

Two mathematicians, Von Neumann and Morgenstern, gave rise to a method more useful for describing individuals behaviour. They postulated in the 1950s that when facing a risky decision people will maximize not their expected value but their expected utility. \cite{Morgenstern53}
The theory is therefore called expected utility theory.
To every possible outcome one assigns a utility value as defined by a real valued utility function. The person does not need to know (and most likely doesn't know) about their utility function. They will implicitly assign utilities and then pick the option that maximizes their expected function value.


\begin{figure}[ht]
\centering
\includegraphics[width=0.5\textwidth]{img/background/riskaversion}%
\caption{Choice from a risk free gift of 30\$ or a lottery where by coin flip you win 10\$ or 50\$. The utility for the risk free choice (point A) is higher than the expected utility of the risky lottery (point B). 
% b) The exponential utility function for different risk parameters $\lambda$. $\lambda < 0$ implies risk seeking, $\lambda = 0 $ risk neutral and $\lambda > 0 $ risk averse behaviour.
}
\label{fig:background:riskaversion}
\end{figure}


To illustrate this we will look at the example illustrated in figure \ref{fig:background:riskaversion}:
A person is asked to choose from a gift of 30\$ or to participate in a fair coin flip where she can win 10\$ or 50\$.
It's easy to see that the expected outcome for both options is 30\$ but the coin flip is risky while the gift is risk free.
The person in figure \ref{fig:background:riskaversion} acts in line with a concave risk function, i.e. the utility grows slower than the outcome. This also leads to a saturation effect where higher values are valued less and less which can often be observed in economics. %https://books.google.de/books?id=II4Nwm1uoCIC&pg=PA156&lpg=PA156&dq=utility+saturation+effect&source=bl&ots=HTlYRm8HtH&sig=jlqdH5UPyURI-SWXYlCJ5DzWXsY&hl=de&sa=X&ved=2ahUKEwjL-cHox8bcAhWPZlAKHVj3DHMQ6AEwAXoECAEQAQ#v=onepage&q=utility%20saturation%20effect&f=false
The utility assigned to the possible outcome of 10\$ is around 40, the utility for 50\$ around 110 (both indicated by the dotted vertical lines). The expected utility of the coin flip (for different unfair coins) is simply the linear interpolation in between these two points. As we are using a fair coin the expected utility is right in the middle marked with the letter B at 75. In comparison the expected utility for the gift of 30\$ is at 100, indicated by letter A. The person in this example would therefore prefer the risk free choice of the gift over the risky coin flip.


\textbf{It is an important observation that concave utility functions describe risk averse decision makers whereas convex utility functions describe risk seeking decision makers. }

Different risk profiles can therefore be modelled by differently curved utility functions. Note that the curvature does not need to remain the same over all outcomes but can change, allowing to model risk seeking behaviour in some realm, e.g. for small amounts of money and risk averse behaviour in others, e.g. loss of life.

A very common and easy to parametrize utility function is the exponential function defined as:
\begin{flalign}
& U(w)  =  
\begin{cases}
	\frac{1-\exp(- \lambda w)}{\lambda} & \lambda \neq 0\\
	w & \lambda = 0
	\label{equ:exp}
\end{cases}
\end{flalign}
It is often chosen for its mathematical convenience yet its applicability is highly questionable as it implies time independent decisions, i.e. past losses are not considered when making new decisions. The impact of the parameter $\lambda$ on the curvature of the function is shown in figure \ref{fig:background:exponential}.

\begin{figure}[ht]
\centering
\includegraphics[width=0.3\textwidth]{img/background/Exponential_Utility_Function.pdf}%
\caption{ 
The exponential utility function for different risk parameters $\lambda$. $\lambda < 0$ implies risk seeking, $\lambda = 0 $ risk neutral and $\lambda > 0 $ risk averse behaviour.
}
\label{fig:background:exponential}
\end{figure}




% ###########################################################################

\subsection{Partially Observable Markov Decision Process}

Markov Decision Processes (MDPs) are a common choice for mathematically modelling risky decision making. Markov Decision Processes are usually defined over a finite state environment in which an agent can take actions with probabilistic outcomes that then affect the environment. At each time step the agent can fully observe in what state the environment is in.

Therefore an MDP is fully defined by the following four items:
\begin{itemize}
    \item Set of states $\mathcal{S}$ (terminal and non-terminal)
    \item Set of actions $\mathcal{A}$
    \item Probabilistic state transitions depending on tuple $\mathcal{S} \times \mathcal{A}$
    \item Reward function $R(s,a)$
\end{itemize}

To model decision making under uncertainty a MDP can be adopted to be partially observable instead. It is then called a Partially Observable Markov Decision Process (POMDP). In contrast to the classical MDP the agent no longer knows which state the environment is in. Instead it gets observations that it must use to infer the environments state.
This means the agent has to work on a probability distribution over states rather than the actual state.

To formally adopt the MDP framework we add two more items:
\begin{itemize}
    \item Observation space $\mathcal{O}$
    \item Observation function $p(o | s, a)$
\end{itemize}


A cartoon connecting all the pieces and showing the interactions between agent and environment can be seen in figure \ref{fig:background:agentenv}. The system is in a cyclical flow alternating between environment and agent. The agent performs an action that causes a state transition in the environment. The new state emits an observation and possibly a reward that are fed back to the agent. The agent uses this information to update its belief state and then acts according to its policy.

\begin{figure}
  \centering
    \includegraphics[width=0.5\textwidth]{img/background/POMDP}
  \caption{Sketch of a partially observable environment. The agent only gets observations and rewards as inputs but cannot see the state of the environment.}
  \label{fig:background:agentenv}
\end{figure}






% add reference: http://www.cassandra.org/arc/papers/aaai94.pdf

\section{Objectives}\label{sec:objectives}
The objective of this research is threefold:
\begin{itemize}
    \item First we want to conduct an experiment to observe peoples behaviour when facing risky choice. 
    \item Second we want to explicitly reassess whether human behaviour is consistently described by the exponential utility function.
    \item Third and finally we want to present an approach to derive utility a persons utility functions based on their observed behaviour
\end{itemize}


\section{Methods}\label{sec:methods}
\subsection{Experiment}

As a way to test our hypothesis we conducted a behavioral experiment to gather empirical data. We developed a web app where people could participate in a simulated investment task.

In the experiment people where told that they own a house that they want to sell. At the beginning of the simulation the housing market is in recession but when the participants wait long enough it will eventually change to a booming market and prices will increase. To introduce pressure on the participants they have to pay maintenance cost for every year they wait.
The goal then is to sell the house as soon as possible after the market has become booming.

The current state of the market is not directly observable by the participant. Instead at each time step one of the following things was shown to them:
\begin{itemize}
    \item Scenario 1: a price for which one of the houses in their neighborhood was sold, samples from a gaussian distribution with different mean depending on the market state.
    \item Scenario 2: the belief of an "expert" (the Bayesian estimate based on the observation history) that shows the chances of the market being in a good state
\end{itemize}

Given this information they had to infer in what state the market actually is and act accordingly.

~\autoref{fig:user-interface} shows how the interface of the experiment looks like for the two scenarios. In the first scenario, the participant was shown a new house price together with all the past observations in the form of a historic plot. In the second scenario they were given only the current belief. Besides this the interface constantly shows the total amount of money that the participant has spent on maintenance so far.

\begin{figure}[!htbp]
    \centering
    \includegraphics[width=0.99\linewidth]{img/methods/experiment_obs_1.png}\\
    \includegraphics[width=0.99\linewidth]{img/methods/experiment_bel_1.png}\\
    \caption{The two experiment scenarios and their respective UI. In (a) subjects are shown the observation history at the top and a new observation at the bottom. In (b) they see the Bayesian estimate based on the observation history.}\label{fig:user-interface}
\end{figure}

~\autoref{fig:states} shows a graphical representation of the underlying environment. The market has three states, \textit{recession} and \textit{booming} and \textit{sold}. The market starts in recession in each experiment and iteratively transitions are performed. At each transition subjects are shown an observation from the current state and can choose between \textit{wait} and \textit{sell}. The experiments ends when subjects sell and they receive a reward based on the last market state.

\begin{figure}[H]
\tikzset{
scale=0.62, every node/.append style={transform shape}
}
\begin {center}
\begin {tikzpicture}[-latex ,auto ,node distance =3cm and 4cm ,on grid ,
semithick , scale=0.5, transform shape,
state/.style ={ circle ,fill=black!20, minimum width =3 cm}]
\node[state] (C){\large Sold};
\node[state] (A) [above left=of C,align=center] {\large Recession};
\node[state] (B) [above right =of C,align=center] {\large Booming};
\coordinate[below of=A] (AA);
\coordinate[below of=B] (BB);
\coordinate[below of=AA] (D);
\coordinate[below of=BB] (E);

\path (A) edge [loop left, line width=1mm, align=center] node[left] {\large wait \\ $\alpha =$ \\  $0.86$} (A);
\path (A) edge [bend left = -25,line width=1mm,align=center] node[below =0.25 cm] {\large sell\\$1.0$} (C);
\path (A) edge [bend left =25,line width=1mm,align=center] node[above] {\large wait\\$0.14$} (B);

\path (B) edge [loop right,line width=1mm,align=center] node[right] {\large wait\\$1.0$} (B);
\path (B) edge [bend right = -25,line width=1mm,align=center] node[below =0.25 cm] {\large sell\\$1.0$} (C);

%\fill[gray!40!white, opacity=0.5] (-6,-1) rectangle (5,6);

\path (A) edge [bend right =25,line width=1mm, dashed] node[left] {\large $Observation$} (D);
\path (B) edge [bend left  =25,line width=1mm, dashed] node[right] {\large $Observation$} (E);
\end{tikzpicture}
\end{center}
    \caption{Market States}\label{fig:states}
\end{figure}

The experiment was conducted with 24 participants and each one of them did the following runs:
\begin{itemize}
    \item 25 runs with random samples from the observations experiment with an easy setup (low maintenance costs.)
    \item 25 runs with random samples from the observations experiment with a hard setup (high maintenance costs.)
    \item 60 runs with random samples from the belief experiment.
\end{itemize}

The participants were paid according to their performance (average reward over all experiment runs) in order to give motivate them for performing as good as possible in the experiment.
% !TEX root = main.tex

\subsection{Solving the risk sensitive POMDP}

%\normalsize
Analyzing the behavior of participants was done by fitting policies based on various utility functions to the behavioral data.
We proceeded in three steps, namely by
a) designing an algorithm for solving a risk-sensitive POMDP (RSPOMDP) using arbitrary utility functions,
b) picking a suitable set of utility functions for evaluation,
c) finding a measure to compare the different policies

\textbf{a) Reinforcement agent design}\\
Marecki \cite{marecki} showed that RSPOMDPs can be solved for arbitrary utility functions using \keyword{reverse value iteration} in Belief Wealth Space.
For this, the original state space must be augmented two times as indicated in figure ~\autoref{fig:augmenation}.
\begin{figure}[H]
\begin {center}
\begin {tikzpicture}[-latex ,auto ,node distance =3cm and 3cm ,on grid,
semithick , state/.style ={ circle ,fill=black!20}]
\node[state] (A) [align=center] {Observation\\Time\\Space};
\node[state] (B) [right of=A,align=center] {Belief\\Time\\Space};
\node[state] (C) [right of=B,align=center] {Belief\\Wealth\\Space};

\path (A) edge [line width=1mm, align=center] node[left] {} (B);
\path (B) edge [line width=1mm,align=center] node[below =0.25 cm] {} (C);
\end{tikzpicture}
\end{center}
\caption{Two augmentations of the original state space are necessary to solve the POMDP for different utility functions.}\label{fig:augmenation}
\end{figure}

The first augmentation transforms the experiment's Observation Time Space into a Belief Time Space. Knowledge of the state-transition probability $\alpha$ (see figure ~\autoref{fig:states}) and the observation distributions are combined using Bayesian inference. Iteratively applied, the resulting \textit{belief} describes at any time the probability of being in the good state:

\begin{flalign}
   &b' = \phi(b,o) := \left( 1 + \frac{\alpha (1-q)}{1 - \alpha + \alpha q} \exp \left( \frac{1}{2} \Delta \right)\right)^{-1}
   \label{equ:belief}
\end{flalign}
with $ \Delta = \left(\frac{o - \mu_2}{\sigma}\right)^2 - \left(\frac{o - \mu_1}{\sigma}\right)^2$ and $\mathcal{N}(\mu_1,\sigma^2)$ and $\mathcal{N}(\mu_2,\sigma^2)$ the observation distributions for the recession and booming state, respectively. This transforms the discrete state POMDP into a continuous MDP.

The second augmentation transforms the Belief Time Space into a Belief Wealth Space. In the case of our experiment this is trivial, as wealth at any time is just the accumulated waiting cost plus either the low or the high reward. Hence the possible wealth values are discrete in the interval 
$\left[\max(\text{Waiting Cost}) + \text{Low Reward}, \text{High Reward}\right]$.

The MDP with the augmented state space can finally be solved by the value iteration algorithm developed in \cite{marecki}:
\begin{flalign*}
    &\text{Initialization:}\\
    &V^{n}_{U}(b,w) = U(w)\\
    &\text{Iteration:}\\
    &V^{n-1}_{U}(b,w) = \max_{a \in A}(\sum_{o \in O}{P(o|b)V^{n}_{U}(\phi(b,o), w + r(a))})
    \label{alg:valiter}
\end{flalign*}
with observations $o$, belief $b$, wealth $w$ and wealth based utility function $U(w)$. The values are initialized with the utility function before the iteration converges them to a stable solution. Basically utility is passed backwards from all possible outcomes through the state space to the beginning of each episode.

\textbf{b) Choice of utility functions}\\
There are infinite many utility functions to choose from. However, the restricted setup of the experiment encourages three categories of selling policies. Policies depending on the belief, on the wealth or on both. For belief dependent policies we used common utility functions from economics, for strictly wealth based policies we derived one.

Policy 1: Selling at a fixed believe\\
A participant or agent sells the house at a belief higher than a fixed threshold. The choice is independent of the current wealth. This behavior is commonly modeled in economics and can be described by the exponential utility function 
\begin{flalign}
& U_{\exp}(w)  =  
\begin{cases}
	\frac{1-\exp(- \lambda w)}{\lambda} & \lambda \neq 0\\
	w & \lambda = 0
	\label{equ:exp2}
\end{cases}
\end{flalign}
Here $\lambda < 0$ models risk-seeking, $\lambda = 0$ risk-neutral and $\lambda > 0$ risk-averse behavior.

Policy 2: Selling at a wealth-dependent believe\\
A participant or agent sells the house at a believe higher than a variable threshold. The threshold is a function of the wealth. This allows different risk-sensitivities for different wealth. 
It can be modeled using the dynamic exponential utility function
\begin{flalign}
& U_{\text{dyn}}(w)  =  
\begin{cases}
	\frac{1-\exp(- \lambda_w w)}{\lambda_w} & \lambda_w \neq 0\\
	w & \lambda_w = 0
	\label{equ:dynexp}
\end{cases}
\end{flalign}
with $\lambda_w$ a function dependent on wealth.

Policy 3: Selling at a fixed wealth\\
A participant or agent sells the house at a fixed wealth, independent of the current believe. This can be modeled with a parametrized $\sinh$ utility function.
\begin{flalign}
	&U_{\sinh}(w) := e^{w_1} - e^{-w_2}
	\label{equ:sinh}
\end{flalign}
where  $w_i = \text{scale}_i \cdot(w+\text{shift}_i)$. Scaling and shifting are parameters that are adjusted to model individual behaviors. 


\textbf{c) Measure for policy comparison}\\
A measure is needed to decide which of the policies fits the behavior of each participant best. We estimated each participant's policy from his or her behavioral data using a Support Vector Machine (SVM) and cross-validation \cite{svm}.
We implemented the SVM such that it separates the belief at the last waiting action and the belief at the selling action with a margin as large as possible.
The estimated policy can then be compared to the candidate policies to establish a best fit.

\section{Results}\label{sec:results}
% * value functions 
% 


\subsection{Value functions in belief wealth space}\label{ssec:val-func}

First, we look at value functions found by using different utility functions in~\autoref{fig:val-func}.
As described in~\autoref{sec:methods} augmented state space encodes the wealth and the belief of the agent. 
Additionally, in our case, time can be related to wealth by the simple formula:

\begin{align*}
\text{time} = \frac{\text{wealth}}{\text{observation\ cost}}
\end{align*}

The policy is the action with maximum return for each belief wealth pair, which is the line that separates the blue (waiting) and the orange (selling) area in~\autoref{fig:val-func}. For example, in the very first plot, the agent always sells at 0.8 belief for any wealth value. We call this line the belief threshold.

In the first row, we present value functions that are found by using the risk-neutral UF (identity function) and an exponential UF. Both functions result in agents simply selling above a certain belief threshold, regardless of the current wealth.
% TODO do we need a policy plot to make this clear?
Exponential UF sets this threshold with the $\lambda$ parameter, which is displayed above the plot.
A negative $\lambda$ induces risk-averse behavior, which results in selling with higher belief than the risk-neutral agent.

The second row shows policies derived by the $\text{sinh}$ UF. As oppesed to the previous policies they are learly time-dependent. We have parametrized $\text{sinh}$ by the scaling and shifting of its input:

\begin{align*}
    U_\text{sinh}(v) := e^{v_1} - e^{-v_2} \text{, where\ } v_i = \text{scale}_i\cdot(v+\text{shift}_i)
\end{align*}

The first plot shows what we call a fixed time policy, where the belief threshold for selling abruptly decreases at a specific wealth value (i.e. time).
The second plot shows a similar policy but with a smoother decrease in belief threshold over the wealth axis.

Lastly, in third row we present the dynamic utility function, where $U$ depends on time by selecting $\lambda(t)$ from a logarithmically spaced sequence. The sequence range is displayed above each plot. Dynamic utility function is hard to parametrize because most picks for $\lambda(t)$ make the resulting function unstable. Furthermore, it is difficult to perform a grid search due to it's many free parameters (possibly as many as time steps.)

\begin{figure}[h]
    \centering
    \includegraphics[width=0.99\linewidth]{img/exp_policy.pdf}\\
    \includegraphics[width=0.99\linewidth]{img/sinh_policy.pdf}\\
    \includegraphics[width=0.99\linewidth]{img/dyn_policy.pdf}
    \caption{Value functions exhibiting different risk-behaviors; from top left: risk neutral agent (utility function is the identity function), risk-seeking agent with exponential utility function, two fixed time agents with different time thresholds, two agents with dynamic exponential utility function.}\label{fig:val-func}
\end{figure}

\subsection{From human behavior to utility functions}\label{ssec:human-behavior}

In~\autoref{ssec:val-func}, we have presented how different utility functions result in different policies.
Now we look at how we derived utility functions and the corresponding risk parameters from behavioral data.
In literature, this is called \keyword{inverse reinforcement learning} \cite{Abbeel2010}.

Inverse reinforcement learning is challenging. Thus, we only try a naive approach, which can be summarized in three steps: 1) observe behavior, 2) estimate policy, and 3) derive utility functions and risk parameters using a grid-search.
The results are plotted in~\autoref{fig:svm_vs_value}, and explained in following. 
First, we get behavioral data for a subject from the expert-setting of the experiment.
For each time step (i.e. year) of each trial, we have a data point with the action subject took, the belief shown to the user and the year. These are plotted in~\autoref{fig:svm_vs_value} with blue for wait action, and orange for sell action.

Second, we estimate the policy using a SVM \cite{svm} with cross-validation. SVM tries to fit a decision boundary, which minimizes the wait actions above and sell actions below this boundary. Here we only take the last two actions into account, i.e. one waiting and one selling actions (plotted with fuller circles), the rest of the actions are plotted in light colors.

The last step is to perform a grid search over the utility functions and their parameters.
Each utility function and its parameters results in a value function from which a policy is derived.
The goal of the grid search is to find the policy that is closest to the SVM decision boundary.

We found $\text{sinh}$ and Exponential UF to work the best. In behavioral data, we observed three kinds of behaviors: 1) people selling at a specific belief threshold, 2) people selling at a specific year, and 3) people lowering their threshold over years. We show an example for each of these behaviors in each row of~\autoref{fig:svm_vs_value}.

\begin{figure}[h]
    \centering
    \includegraphics[width=0.99\linewidth]{img/fit}
    \caption{Examples from three human behaviors distinctly observed and reproduced with RL agents. Behaviors by row: 1) Constant belief threshold, modeled by the risk-averse exponential utility function, 2) Fixed time threshold, modeled by sinh utility, 3) Mixed strategy, modeled by sinh exponential utility. The left column shows the empirical optimal split of data using linear kernel support vector machine (with cross-validation). The right column shows optimal policy that is closest to the SVM split, found via grid-search.}
    \label{fig:svm_vs_value}
\end{figure}

\subsection{Analysis of agents}
After reproducing the human behavior with agents, we can compare the performance of different agents.
We sampled 10.000 episodes with each agent and present their performance in~\autoref{fig:perf}. The accumulated reward of each agent is presented with a box plot, where the green line shows the median of the accumulated reward, the box shows the Q1-Q3 quartile, and the circles show the outliers\footnote{Refer to matplotlib documentation and Wikipedia for an exact description of a box plot: \url{https://matplotlib.org/api/_as_gen/matplotlib.pyplot.boxplot.html}, \url{https://en.wikipedia.org/wiki/Quartile}}.
The ratio of sells in the good state is shown with a bar plot.
Here, bars show the ratio of selling in the good state and the black lines show the variance.

The risk-averse agent has lower variance both on average reward, and the ratio of selling in the good state. However, it has a similar number of outliers in accumulated reward as the risk-neutral agent, which can be easily explained by the high selling threshold that results in a longer waiting time. The agents using sinh UF, on the other hand, have no outliers. And the mixed strategy has a similar median value with slightly larger quartile.
This shows the benefit of risk modeling, even in such simple scenarios.

\begin{figure}[h]
    \centering
    \includegraphics[width=0.99\linewidth]{img/performance.pdf}
    \caption{Accumulated reward and ratio of selling in the good state.}\label{fig:perf}
\end{figure}

\section{Discussion}\label{sec:discussion}
Using the experimental data and reinforcement learning agents, we could provide evidence that the exponential utility function is not able to capture all human risk behaviour under uncertainty. While the EUF implies a time-independent policy, we found that the majority of our participants acted in a time-dependent manner. The parametrized $\text{sinh}$ utility function, on the other hand, was able to model these participants fairly well.

However, much work is left to be done both in the theoretical and the empirical realm. 
For the theoretical part effort is needed to find a way to solve the inverse reinforcement learning problem more directly, i.e. to obtain a utility function directly from behavioural data without having to search a vast parameter space.
Furthermore, online methods like Q-learning can be explored to model human behavior better by considering the order of the episodes and the learning process.

For the empirical part it would be interesting to conduct a reversed experiment where the house prices increase over time until the market transitions into a state of recession. This could show whether the observed risk seeking behaviour of many participants was only induced by high waiting costs. 

% Acknowledgements should only appear in the accepted version.
\section*{Acknowledgements}

This work was supported by the NI Department. The authors gratefully acknowledge the helpful discussions and technical assistance provided by Rong Guo, Vaios Laschos and Robert Seidel.
We also thank the WZB department for assisting us during the execution of the experiment.


% In the unusual situation where you want a paper to appear in the
% references without citing it in the main text, use \nocite
\nocite{langley00}

\bibliography{example_paper}
\bibliographystyle{icml2018}


%%%%%%%%%%%%%%%%%%%%%%%%%%%%%%%%%%%%%%%%%%%%%%%%%%%%%%%%%%%%%%%%%%%%%%%%%%%%%%%
%%%%%%%%%%%%%%%%%%%%%%%%%%%%%%%%%%%%%%%%%%%%%%%%%%%%%%%%%%%%%%%%%%%%%%%%%%%%%%%
% DELETE THIS PART. DO NOT PLACE CONTENT AFTER THE REFERENCES!
%%%%%%%%%%%%%%%%%%%%%%%%%%%%%%%%%%%%%%%%%%%%%%%%%%%%%%%%%%%%%%%%%%%%%%%%%%%%%%%
%%%%%%%%%%%%%%%%%%%%%%%%%%%%%%%%%%%%%%%%%%%%%%%%%%%%%%%%%%%%%%%%%%%%%%%%%%%%%%%
\appendix
\section{Behavioral data for all subjects}
All of the behavioral data fitted with linear kernel support vector machine is shown in~\autoref{fig:all_beh}.


\begin{figure}
    \centering
    \includegraphics[width=0.99\linewidth]{img/all_peeps.pdf}
    \caption{All behavioral data collected and fitted with a linear kernel SVM. Participant 2 and 7 have been removed because of invalid behavior (they always sold at very first time step, which by definition is never booming.)}
    \label{fig:all_beh}
\end{figure}

\end{document}


% This document was modified from the file originally made available by
% Pat Langley and Andrea Danyluk for ICML-2K. This version was created
% by Iain Murray in 2018. It was modified from a version from Dan Roy in
% 2017, which was based on a version from Lise Getoor and Tobias
% Scheffer, which was slightly modified from the 2010 version by
% Thorsten Joachims & Johannes Fuernkranz, slightly modified from the
% 2009 version by Kiri Wagstaff and Sam Roweis's 2008 version, which is
% slightly modified from Prasad Tadepalli's 2007 version which is a
% lightly changed version of the previous year's version by Andrew
% Moore, which was in turn edited from those of Kristian Kersting and
% Codrina Lauth. Alex Smola contributed to the algorithmic style files.
