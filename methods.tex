% !TEX root = main.tex

\subsection{Solving the risk sensitive POMDP}

\normalsize
Analyzing the behavior of participants was done by fitting policies based on various utility functions to the behavioral data..
We proceeded in three steps, namely by,
a) designing a reinforcement agent that can model arbitrary utility functions on the experiment, 
b) picking a \enquote{suitable} subset of policies from the infinite space of utility functions for evaluation,
c) finding a measure to compare the applicability of different policies
>>>>>>> Stashed changes

\textbf{a) Reinforcement agent design}\\
Marecki \cite{marecki} showed that RSPOMDPs can be solved for arbitrary utility functions using \keyword{reverse value iteration} in Belief Wealth Space.
For this, the original state space must be augmented two times as indicated in figure ~\autoref{fig:augmenation}
\begin{figure}[H]
\begin {center}
\begin {tikzpicture}[-latex ,auto ,node distance =3cm and 3cm ,on grid,
semithick , state/.style ={ circle ,fill=black!20}]
\node[state] (A) [align=center] {Observation\\Time\\Space};
\node[state] (B) [right of=A,align=center] {Belief\\Time\\Space};
\node[state] (C) [right of=B,align=center] {Belief\\Wealth\\Space};

\path (A) edge [line width=1mm, align=center] node[left] {} (B);
\path (B) edge [line width=1mm,align=center] node[below =0.25 cm] {} (C);
\end{tikzpicture}
\end{center}
\caption{Two augmentations of the original state space are necessary to }\label{fig:augmenation}
\end{figure}

The first augmentation transforms the given observation-time-space into a belief-time-space. Knowledge of the state-transition probability $\alpha$ and the observation distributions are combined using bayesian inference. Iteratively applied, the resulting \textit{belief} describes at any time the probability of being in the good state (see ~\autoref{equ:belief}). 
%The POMDP is thereby transformed into a MDP.
\begin{flalign}
   &b' = \phi(b,o) := \left( 1 + \frac{\alpha (1-q)}{1 - \alpha + \alpha q} \exp \left( \frac{1}{2} \Delta o \right)\right)^{-1}
   \label{equ:belief}
\end{flalign}
where $ \Delta o = \left(\frac{o - \mu_2}{\sigma}\right)^2 - \left(\frac{o - \mu_1}{\sigma}\right)^2$.

NOT READY from here
The second augmentation transforms the belief-time-space into a belief-wealth-space. This is done by ...
Solve the MDP with value iteration on an augmented state space. \cite{bauerle}
\begin{flalign}
    V^{n}_{U}(b,w) &= U(w)\tag{Initalization} \\
    V^{n-1}_{U}(b,w) &= \max_{a \in A}(\sum_{o \in O}{P(o|b)V^{n}_{U}(b', w + r(a))}) \tag{Iteration}
\end{flalign}
where $b \in [0,1]$ was discretized into ABCD points and w is discrete by nature of the experiment. 
NOT READY till here

\textbf{b) Choice of utility functions}\\
The space of utility functions is infinite. However, the restricted setup of the experiment (the believe-wealth state space) allow three main categories of selling strategies (policies).
Those are either policies depending on the believe, or on the wealth or on both. 
(FOOTNOTE: we chose three strategies that are easy to model. not comprehensive. other strategies possible. but sufficient for evaluating our hypothesis.)

Policy 1: Selling at a fixed believe\\
A participant or agent sells the house at a believe higher than a fixed threshold $\in [0,1]$, independent of the current wealth. This behavior is a common model in economic and described by the exponential utility function 
\begin{flalign}
& U_{\exp}(w)  =  
\begin{cases}
	\left(1-exp(- \lambda w)\right) / \lambda & \lambda \neq 0\\
	w & \lambda = 0
\end{cases}
\end{flalign}
Here $\lambda > 0$ models risk-seeking behavior, $\lambda = 0$ risk-neutral behavior and $\lambda < 0$ risk-averse behavior.

Policy 2: Selling at a wealth-dependent believe\\
A participant or agent sells the house when the believe $b$ crosses a threshold $\in [0,1]$ with the threshold dependent on the current wealth. 
This can be modeled using the dynamic exponential utility function
\begin{flalign}
& U_{\text{dyn}}(w)  =  
\begin{cases}
	\left(1-exp(- \lambda_w w)\right) / \lambda_w & \lambda_w \neq 0\\
	w & \lambda_w = 0
\end{cases}
\end{flalign}
where $\lambda_w$ is a function dependent on wealth.

Policy 3: Selling at a fixed wealth
A participant or agent sells the house at a fixed wealth, independent of the current believe. This can be modeled with a parametrized $\sinh$ utility function.
\begin{flalign}
	U_{\sinh}(w) := e^{w_1} - e^{-w_2}
\end{flalign}
where  $w_i = \text{scale}_i \cdot(w+\text{shift}_i)$. Scaling and shifting are parameters that are adjusted to model individual strategies. 


\textbf{c) Measure for policy comparison}\\
A measure is needed to decide which of the candidate policies fits the behavior of each participant best. We estimated each participants policy from the behavioral using a Support Vector Machine (SVM)  with cross-validation. The SVM separates the belief at the last waiting action and the belief at the selling action with a margin as large as possible, where we allow XXX percent as outliers. We then compare the estimated policy to the candidate policies to establish a best fit. We argue that this method is sufficient to evaluate the hypothesis of this paper.




