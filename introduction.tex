In our everyday lives we are forced to act under uncertainty, meaning in circumstances where we do not have full information. Imagine for example that you want to catch the local bus to university. You know when it is supposed to arrive, but it could very well be that the bus is a couple of minutes early or late - there is uncertainty involved. How we act under uncertainty is an indicator of our risk profile: Some people will want to avoid the risk of missing the bus and decide to leave for the station rather early, others might speculate that the bus will be late and leave accordingly, expressing a more risky behaviour. This behaviour is called the risk profile or risk-sensitivity of a person.\\
It has long been a goal of economics to capture risk profiles in a consistent manner. While it's easy to simulate behaviour for a given policy the reverse problem is yet to be solved.
In this paper we will show a machine learning approach that can be used to bridge this gap from observing peoples behaviour and estimating their utility function.
Based on this approach we will reassess the hypothesis that the so called exponential utility function models humans behaviour.